% Template by ASP
\documentclass[letterpaper]{article}
\usepackage[utf8]{inputenc}
\usepackage[T1]{fontenc}
\usepackage[english]{babel}
\usepackage[top=3cm, bottom=3cm, left=3.85cm, right=3.85cm]{geometry}
\usepackage[onehalfspacing]{setspace}
\usepackage{amsmath,amssymb,amsthm,wasysym}
\usepackage{graphicx,mathptmx}
\usepackage[usenames,dvipsnames]{color}
\usepackage{hyperref}

\newtheoremstyle{xxx}
  {0pt}{1pt}
  {\color{BrickRed}}
  {0pt}
  {\bf}{.}{ }{}
\newtheoremstyle{yyy}
  {0pt}{1pt}
  {\color{OliveGreen}}
  {0pt}
  {\bf}{.}{ }{}
\newtheoremstyle{zzz}
  {0pt}{1pt}
  {\color{Blue}}
  {0pt}
  {\bf}{.}{ }{}
\newtheoremstyle{evil}
  {2pt}{2pt}
  {\color{red} \bf}
  {0pt}
  {\bf}{.}{ }{}

\theoremstyle{xxx}
\newtheorem{thrm}{Theorem}
\newtheorem{prop}{Proposition}
\theoremstyle{evil}
\newtheorem{warn}{Warning}
\newtheorem{evil}{Evil}
\theoremstyle{yyy}
\newtheorem{corl}{Corollary}
\newtheorem{lmma}{Lemma}
\theoremstyle{plain}
\newtheorem{expl}{Example}
\theoremstyle{zzz}
\newtheorem{defn}{Definition}

\newcommand{\notn}{\paragraph{Notation.}}
\newcommand{\recl}{\paragraph{Recall.}}
\newcommand{\rmrk}{\paragraph{Remark.}}
\newcommand{\note}{\paragraph{Note.}}

\begin{document}

% Two-column title block
\begin{minipage}[b]{0.7\linewidth}
{\huge ASICs and Decentralization FAQ}
\end{minipage}
\begin{minipage}[b]{0.3\linewidth}
  \begin{flushright}
    andytoshi{\tiny \\
    \tiny 1m5JtHuGgWYaFV7K6mSCnpzVVmheqoZMg}
  \end{flushright}
\end{minipage}
\\

% Actual content start
\section{What is Proof of Work?}
\begin{enumerate}
\item \textbf{What is Proof of Work?}

As it applies to Bitcoin, a \emph{proof of work} is a computational proof
that some scarce resource was consumed. Such a proof is possible because
it appears that computational resources are physically bounded by available
time, space and energy\footnote{See, for example, \url{http://arxiv.org/abs/quant-ph/0502072}
for a discussion on the emergence of fundamental computational limits from
physics.}

Further, a proof of work commits to some data, effectively "signing" it
with some consumed resource.

\item \textbf{How and why is it used?}

Such proofs are useful because they ensure malicious actors are limited
in their capacity to produce valid proofs of alternate histories, giving
the remaining actors sufficient time and bandwidth to come to consensus
on a single history, as long as they have more resources than the malicious
ones.

Bitcoin's main innovation was aligning economic incentives to discourage
resource-rich actors from becoming malicious.

Here a ``history'' is formed by a chain of proofs-of-work, each of which
signs (a) a collection of valid Bitcoin transactions, and (b) the previous
proof-of-work. Users are able to achieve consensus not only of the transactions,
but of their order in time.

\item \textbf{How does the proof-of-work affect consensus?}

In order that all actors (including those not ``mining'' or generating proofs
of work) can quickly reach consensus, the proof-of-work must be quickly
verifiable. The greater the ratio of generation time to verification time,
the better.

In order than all actors can reach consensus, the proof-of-work must require
enough time that previously proven transaction history can propagate and be
verified before the history is extended. Here ``enough time'' depends on
both on the amount of data to be verified (the blocksize) as well as the
speed at which the proof-of-work itself can be verified.

\item \textbf{How does the proof-of-work affect decentralization?}

In order that individual actors do not gain a disproportionate advantage ---
so that in the economic limit one actor has all the mining power and
leases it out, obviating the proof-of-work since this is functionally
identical (but ecologically trillions of trillions of times worse) to
this actor simply digitally signing each transaction --- the proof-of-work
must be completely parallelizable and require no state. That is, one actor
with $2N$ hashing power should have the same return as two actors each with
$N$ hashing power.

Bitcoin achieves this property by using a proof-of-work based on finding
partial hash preimages. The work is accomplished by repeated hashing, and
each attempt has a tiny i.i.d. chance of success (this is derived from the
so-called \emph{random oracle assumption} of the specific hash algorithm).
This gives rise to a Poisson process, which is well-understood but in some
ways unintuitive.

Another crucial way that the proof-of-work affects decentralization is
in its physical attributes, which we cover in the next section.
\end{enumerate}

\section{The physics of Proof-of-Work.}

\begin{enumerate}

\item \textbf{What is (ir)reversible computing and why does it matter?}

As this is not a physics paper, this section necessarily contains
several claims with neither detail nor justification. The curious
reader is encouraged to read the Wikipidia article
\footnote{\url{https://en.wikipedia.org/wiki/Reversible_computing}}
and its references.

Reversible, or adiabatic computing, is a term for computing without
increase in entropy. Such computations are reversible in time and
therefore need to be injective as functions of their input, so a
hash-based proof-of-work cannot be completely reversible. However,
some components of the hash function may be reversibly computable,
which is useful because reversible computations require no energy
--- more concretely, in the limit as computational speed goes to
zero, the energy requirement of reversible computations also goes
to zero.

Ordinary, non-reversible computations are subject to Landauer's
principle\footnote{\url{https://en.wikipedia.org/wiki/Landauer\%27s_principle}}
which provides a lower bound on the energy required to flip a
single bit.

What this means is that ultimately, any irreversible proof of
computational work is also a proof of physical work, \emph{i.e.}
energy dissipation.

\item \textbf{What is the thermodynamic limit?}

The thermodynamic limit describes the state at which proof-of-work
is actually done at Landauer's lower bound for required energy
dissipation. At this point, we are actually better off than the
requirement that a single $2N$-powered actor has the same advantage
as two $N$-powered actors, because now $N$ is proportional not only
to hashing speed but also to heat dissipation requirements, and it
is easier for two physically-separated actors to dissipate heat than
for just one.

Therefore, in the thermodynamic limit we have a physical incentive
for decentralization.

Fortunately, because the thermodynamic limit is also the limit of
energy efficiency, Bitcoin's incentives for efficient proof-of-work
are also incentives for miners to push toward the thermodynamic
limit.
\end{enumerate}

\section{Actual frequently asked questions.}

\begin{enumerate}
\item \textbf{Are ASIC's evil?}

No, dedicated hardware brings us closer to the thermodynamic limit,
and is therefore eventually a good thing for mining decentralization.
Also, because ASIC's produce more hashes for the same amount of
energy, they produce stronger proofs-of-work with proportionally less
environmental impact.

However, ASIC's bring with them a risk of manufacturer centralization,
such as what we currently see with Bitcoin. Market forces will
eventually break this monopoly, but one thing speeding up the process
is that Bitcoin uses the \texttt{SHA2} hashing algorithm, which was
designed for easy development of dedicated hardware. Therefore,
relatively little startup capital is needed to develop Bitcoin ASIC's.

Further, regardless of one's personal feeling toward ASIC's, they
are inevitable. Dedicated hardware will always be more efficient
than general-purpose hardware (exactly because it is closer to the
thermodynamic limit) and Bitcoin's incentives are aligned for
ever-increasing efficiency.

\item \textbf{Is ASIC resistance desirable?}

No. ASIC resistance typically involves increasing algorithmic
complexity to discourage ASIC developers. However, ASIC's are
still inevitable; all ASIC resistance does is increase the
startup capital required and therefore increase centralization
of manufacturing.

Further, increasing the complexity of proof generation often means
also increasing the complexity of proof validation, often
disproportionately slow. This discourages (unpaid) non-mining
validators, which also increases centralization.

\item \textbf{Is ASIC resistance possible?}

ASIC resistance, in the sense of making life difficult for ASIC
manufacturers (and therefore reducing the number of distinct
manufacturers) is possible. But it is halting-complete to create
an algorithm which runs at the same speed on general-purpose and
dedicated hardware, so ultimately ASIC resistance is futile.

(Schemes such as ``the developers will just change the proof-of-work
algorithm if ASIC's appear'' do not even make sense --- in a decentralized
currency the developers have no such power, while in a centralized
currency proof-of-work is a completely unnecessary waste of power.)

\item \textbf{Is memory hardness desirable?}

No. Memory hardness has the effect of increasing ASIC board footprint,
weakening the heat-dissipation decentralization provided by the
thermodynamic limit. Further, it increases the capital costs of
mining equipment relative to the energy costs, which also encourages
centralization (since established miners have amortized their equipment
more than new miners). These effects are amplified by the fact that
SRAM is both several times faster and several times more expensive
than DRAM.

Also, memory hard proofs-of-work often require lots of memory on the
part of the verifiers\footnote{Though not necessarily --- see for example
John Tromp's ``cuckoo cycle'' scheme at \url{https://bitcointalk.org/index.php?topic=451177.0}.},
which is bad for decentralization as already discussed.

As an aside, since memory is far away and expensive to access on general
purpose computers, memory hardness actually increases the benefit provided
by ASIC's! This is contrary to the goals of most memory-hard advocates,
and as we have seen above, memory-hardness worsens the centralizing effects
of ASIC's while weakening the decentralizing effects\footnote{In fact, at
the time of this writing, \texttt{scrypt} ASIC's are just appearing on the
market, and do indeed provide a proportionally greater gain in hashpower
than did \texttt{SHA2} ASIC's.}.

\item \textbf{Is \texttt{scrypt} better than \texttt{SHA2}?}

No, for a few reasons:
\begin{enumerate}
\item It is much slower to validate, reducing scalability and discouraging
non-mining validation.
\item It is a more complex ``ASIC-resistant'' algorithm. See above.
\item It is more memory-hard than \texttt{SHA2} (though depending on parameters,
it can hardly be considered ``memory-hard'' --- see Litecoin's settings, for
example). See above.
\end{enumerate}

\item \textbf{Is Primecoin better than \texttt{SHA2}?}
No, for a few reasons:
\begin{enumerate}
\item It is much slower to validate, reducing scalability and discouraging
non-mining validation.
\item It is a more complex ``ASIC-resistant'' algorithm. See above.
\item It is an ad-hoc algorithm with no reason to believe that a $2N$-powered
party has the same advantage as an $N$-powered party.
\end{enumerate}

\item \textbf{What about ``useful'' proofs-of-work?}

These are typically bad ideas for all the same reasons that Primecoin is,
and also bad for a new reason: from the network's perspective, the purpose
of mining is to secure the currency, but from the miner's perspective, the
purpose of mining is to gain the block reward. These two motivations
complement each other, since a block reward is worth more in a secure
currency than in a sham one, so the miner is incentivized to secure the
network rather than attacking it.

However, if the miner is motivated not by the block reward, but by some
social or scientific purpose related to the proof-of-work evaluation,
then these incentives are no longer aligned (and may in fact be opposed,
if the miner wants to discourage others from encroaching on his work),
weakening the security of the network.

\end{enumerate}

\end{document}


